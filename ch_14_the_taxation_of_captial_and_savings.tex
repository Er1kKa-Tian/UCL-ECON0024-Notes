\chapter{The Taxation of Capital and Savings}

\fancyhead[L]{ECON0024}
\fancyhead[C]{Ch.14 The Taxation of Capital and Savings}
\fancyhead[R]{Xiaotian Tian}
\fancyfoot[L]{\hyperlink{tableofcontents}{Back to Table of Contents}}
\fancyfoot[R]{Xiaotian Tian}

\section{Guiding Principles for Taxation of Savings}


\section{$\star$ Optimal and Non-Optimal Taxation in An Inter-temporal Context}

    \subsection{Simple Two-Period Model without Uncertainty}

        \subsubsection{Setup}

            Consider a 2 periods model where:
            \begin{itemize}
                \item An individual receives a fixed income/endowment $Y_1$ in period 1, and allocates it between consumption in period 1 $C_1$ and consumption in period 2 $C_2$
                \item Savings $Y_1-C_1$ earn a risk-free rate of return $r$, and all the payout is consumed in period 2
                \item All individuals can borrow or lend at this exogenous risk-free interest rate $r$
                \item The individual only cares about consumption with a discount rate $\rho$
            \end{itemize}

        \subsubsection{Optimisation with No Tax}

            In absence of taxation, the individual solves this maximisation:
            
            \begin{maxi}|s|{C_1,C_2}{V=U(C_1) + \frac{1}{1+\rho} \cdot U(C_2)}{\label{eqn:tax_intertemp_opt}}{}
                \addConstraint{C_2}{=(1+r)\cdot (Y_1-C_1)}
            \end{maxi}

            This can be written as:
            
            \begin{equation*}
                \max_{C_1} V = U(C_1) + \frac{1}{1+\rho} \cdot U\big( (1+r)\cdot (Y_1-C_1) \big)
            \end{equation*}

            Solving this, we obtain the first order condition and the familiar \emphb{Euler equation} for intertemporal allocation of consumption:

            \begin{equation}
                \color{red}
                \frac{\partial V}{\partial C_1} = \frac{\partial U}{\partial C_1} - \frac{1+r}{1+\rho} \cdot \frac{\partial U}{\partial C_2} = 0
                \iff
                \text{IMRS} = \frac{\frac{\partial U}{\partial C_1}}{\frac{\partial U}{\partial C_2}} = \frac{1+r}{1+\rho}
                \label{eqn:tax_intertemp_opt_result}
            \end{equation}

    \subsection{Pure Income Tax (Not Optimal)}

        Suppose there is a tax on income at a constant rate $t$.

        Then, the individual has to pay a tax of $tY_1$ on the endowment received in period 1 and a tax of $t \cdot r \cdot [(1-t)Y_1 - C_1]$ on the interest income received in period 2 (the principal is not taxed, only interest is taxed).

        The optimisation problem becomes:

        \begin{maxi}|s|{C_1,C_2}{V=U(C_1) + \frac{1}{1+\rho} \cdot U(C_2)}{\label{eqn:tax_intertemp_opt_income_tax}}{}
            \addConstraint{C_2}{=[1+(1-t)r]\cdot [(1-t)Y_1-C_1]}
        \end{maxi}

        Solving this, we will get a different FOC and a different IMRS:

        \begin{equation}
            \color{red}
            \frac{\partial V}{\partial C_1} = \frac{\partial U}{\partial C_1} - \frac{1+(1-t)r}{1+\rho} \cdot \frac{\partial U}{\partial C_2} = 0
            \iff
            \text{IMRS} = \frac{\frac{\partial U}{\partial C_1}}{\frac{\partial U}{\partial C_2}} = \frac{1+(1-t)r}{1+\rho}
            \label{eqn:tax_intertemp_opt_income_tax_result}
        \end{equation}

        Different IMRS in equation \ref{eqn:tax_intertemp_opt_result} and \ref{eqn:tax_intertemp_opt_income_tax_result} indicates that, \empha{as a pure income tax incorporates tax on capital earnings, the rate of return deceases, distorting the inter-temporal allocation of consumption}.

    \subsection{Pure Consumption Tax / VAT (Optimal)}

        Suppose there is 

    \subsection{Income Tax with Interest Exemption (Optimal)}

    \subsection{Tax on Uncertain Returns / Tax on Capital Returns with Risk-free Return Exemption (Optimal)}

    \subsection{Key Arguments for Taxing the Return to Savings}